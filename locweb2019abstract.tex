%
% The first command in your LaTeX source must be the \documentclass command.
\documentclass[sigconf]{acmart}

%
% defining the \BibTeX command - from Oren Patashnik's original BibTeX documentation.
\def\BibTeX{{\rm B\kern-.05em{\sc i\kern-.025em b}\kern-.08emT\kern-.1667em\lower.7ex\hbox{E}\kern-.125emX}}
    
% Rights management information. 
% This information is sent to you when you complete the rights form.
% These commands have SAMPLE values in them; it is your responsibility as an author to replace
% the commands and values with those provided to you when you complete the rights form.
%
% These commands are for a PROCEEDINGS abstract or paper.
\copyrightyear{2019}
\acmYear{2019} 
\setcopyright{iw3c2w3}
\acmConference[WWW '19 Companion]{Companion Proceedings of the 2019 World Wide Web Conference}{May 13--17, 2019}{San Francisco, CA, USA}
\acmBooktitle{Companion Proceedings of the 2019 World Wide Web Conference (WWW '19 Companion), May 13--17, 2019, San Francisco, CA, USA}
\acmPrice{}
\acmDOI{10.1145/3308560.3316539}
\acmISBN{978-1-4503-6675-5/19/05}


%
% These commands are for a JOURNAL article.
%\setcopyright{acmcopyright}
%\acmJournal{TOG}
%\acmYear{2018}\acmVolume{37}\acmNumber{4}\acmArticle{111}\acmMonth{8}
%\acmDOI{10.1145/1122445.1122456}

%
% Submission ID. 
% Use this when submitting an article to a sponsored event. You'll receive a unique submission ID from the organizers
% of the event, and this ID should be used as the parameter to this command.
%\acmSubmissionID{123-A56-BU3}

%
% The majority of ACM publications use numbered citations and references. If you are preparing content for an event
% sponsored by ACM SIGGRAPH, you must use the "author year" style of citations and references. Uncommenting
% the next command will enable that style.
%\citestyle{acmauthoryear}

%
% end of the preamble, start of the body of the document source.
\begin{document}

%
% The "title" command has an optional parameter, allowing the author to define a "short title" to be used in page headers.
\title{Indoor Towers, DPIs, and More People in Parks at
  Night:\\New Trends in Mobile Phone Location Research}

%
% The "author" command and its associated commands are used to define the authors and their affiliations.
% Of note is the shared affiliation of the first two authors, and the "authornote" and "authornotemark" commands
% used to denote shared contribution to the research.
\author{Leo Ferres}
%\authornote{Both authors contributed equally to this research.}
\email{lferres@udd.cl}
\orcid{0000-0002-5899-9051}
\affiliation{%
  \institution{Data Science Institute\\Faculty of
  Engineering, Universidad del Desarrollo\\Telef\'onica
  R\&D}
%  \streetaddress{P.O. Box 1212}
  \city{Santiago}
  \state{Chile}
%  \postcode{43017-6221}
}

% \affiliation{%
%   \institution{{Data Science Institute, Faculty of
%   Engineering\\Universidad del Desarrollo} \affil[Telef\'onica
%   R\&D}
%   \streetaddress{P.O. Box 1212}
%   \city{Dublin}
%   \state{Ohio}
%   \postcode{43017-6221}
% }

%
% By default, the full list of authors will be used in the page headers. Often, this list is too long, and will overlap
% other information printed in the page headers. This command allows the author to define a more concise list
% of authors' names for this purpose.
%\renewcommand{\shortauthors}{Trovato and Tobin, et al.}

%
% The abstract is a short summary of the work to be presented in the article.
\begin{abstract}
  In this keynote, I will present some results we have obtained using
  location data from mobile phones interacting with information (news)
  websites, mobile apps like Pokemon GO and Twitter, and large
  physical spaces like shopping malls. I will draw some conclusions
  and generally discuss about the properties of mobile phone data for
  location-based research to finally close with some remarks about
  privacy and data security.
\end{abstract}

%
% The code below is generated by the tool at http://dl.acm.org/ccs.cfm.
% Please copy and paste the code instead of the example below.
\begin{CCSXML}
<ccs2012>
<concept>
<concept_id>10003120.10003130.10011762</concept_id>
<concept_desc>Human-centered computing~Empirical studies in collaborative and social computing</concept_desc>
<concept_significance>300</concept_significance>
</concept>
</ccs2012>
\end{CCSXML}

\ccsdesc[300]{Human-centered computing~Empirical studies in collaborative and social computing}

%
% \begin{CCSXML}
% <ccs2012>
%  <concept>
%   <concept_id>10010520.10010553.10010562</concept_id>
%   <concept_desc>Computer systems organization~Embedded systems</concept_desc>
%   <concept_significance>500</concept_significance>
%  </concept>
%  <concept>
%   <concept_id>10010520.10010575.10010755</concept_id>
%   <concept_desc>Computer systems organization~Redundancy</concept_desc>
%   <concept_significance>300</concept_significance>
%  </concept>
%  <concept>
%   <concept_id>10010520.10010553.10010554</concept_id>
%   <concept_desc>Computer systems organization~Robotics</concept_desc>
%   <concept_significance>100</concept_significance>
%  </concept>
%  <concept>
%   <concept_id>10003033.10003083.10003095</concept_id>
%   <concept_desc>Networks~Network reliability</concept_desc>
%   <concept_significance>100</concept_significance>
%  </concept>
% </ccs2012>
% \end{CCSXML}

% \ccsdesc[500]{Computer systems organization~Embedded systems}
% \ccsdesc[300]{Computer systems organization~Redundancy}
% \ccsdesc{Computer systems organization~Robotics}
% \ccsdesc[100]{Networks~Network reliability}

%
% Keywords. The author(s) should pick words that accurately describe the work being
% presented. Separate the keywords with commas.
\keywords{Mobile phones, CDR and XDR datasets, deep packet inspection,
  data science for social good, location data mining}

%
% A "teaser" image appears between the author and affiliation information and the body 
% of the document, and typically spans the page. 
% \begin{teaserfigure}
%   \includegraphics[width=\textwidth]{sampleteaser}
%   \caption{Seattle Mariners at Spring Training, 2010.}
%   \Description{Enjoying the baseball game from the third-base seats. Ichiro Suzuki preparing to bat.}
%   \label{fig:teaser}
% \end{teaserfigure}

%
% This command processes the author and affiliation and title information and builds
% the first part of the formatted document.
\maketitle

\section{Introduction}
We're more connected than ever. In part, this is due to the high
availability (in both developed and developing countries) of
relatively cheap smart phones that we carry with us all the time. In
fact, in 2018, more than 52\% of the whole website traffic was
generated by mobile phones. Unlike desktops or laptops, one defining
characteristic of mobile phones is that they are always geo-located,
either by GPS, triangulation, or antenna connections. Thus, mobile
phone data sets, either Call Detail Records (CDRs) or Data Detail
Records (XDRs) constitute a potential treasure trove of
information about what people do in the physical world, not only when
interacting with it, but also when accessing information online.

\section{Data and studies}
\label{sec:findings}

Through our association to Telef\'onica Chile R\&D department, we have
been privileged in our access to mobile phone data sets to study
socially relevant issues. In this talk, I will present results from
several studies done using ecologically-valid, very large data sets of
billions of mobile phone usage data (CDRs, XDRs, and a bit of an even
lower level of analyses, deep packet inspection) and the towers they
connect to, drawing conclusions and predicting different kinds of
behaviors, from social mixing \cite{Beiro2018}, to news consumption,
to gender equality. All these studies were conducted by analyzing web
traffic either by proxy to applications like Pokemon Go
\cite{Graells-Garrido2017}, or effectively through DNS resolution as
in the news study or as in the apps study
\cite{Graells-Garrido:2018:WMA:3184558.3191561}. I will conclude by
talking about the coming trends in the field of mobile phone data
analysis, its limitations, and spend some time discussing issues of
privacy, data security, anonymization and general data responsibility
for researchers and the company providing the data.

\section{Bio}

Leo Ferres is an associate professor of computer science at the Data
Science Institute, Universidad del Desarrollo in Santiago, Chile, and
a Fellow of Telef\'onica Research \& Development, also in Santiago. He
has published widely in high-performance computing, computational
social science, using mobile and telephony data for social good, and
data mining in general. Together with UNICEF, the ISI Foundation, and
The GovLab, he is now investigating issues pertaining to gender and
gender gaps, social inclusion at the city scale, and news diffusion
using mobile phone data.

% \section{Appendices}

% If your work needs an appendix, add it before the ``\verb|\end{document}|'' command at the conclusion of your source document. 

% Start the appendix with the ``\verb|appendix|'' command:
% \begin{verbatim}
%   \appendix
% \end{verbatim}
% and note that in the appendix, sections are lettered, not numbered. This document has two appendices, demonstrating the section and subsection identification method.

% \section{SIGCHI Extended Abstracts}

% The ``\verb|sigchi-a|'' template style (available only in \LaTeX\ and not in Word) produces a landscape-orientation formatted article, with a wide left margin. Three environments are available for use with the ``\verb|sigchi-a|'' template style, and produce formatted output in the margin:
% \begin{itemize}
% \item {\verb|sidebar|}:  Place formatted text in the margin.
% \item {\verb|marginfigure|}: Place a figure in the margin.
% \item {\verb|margintable|}: Place a table in the margin.
% \end{itemize}

%
% The acknowledgments section is defined using the "acks" environment (and NOT an unnumbered section). This ensures
% the proper identification of the section in the article metadata, and the consistent spelling of the heading.
\begin{acks}
  Grateful to Movistar-Telef\'onica Chile, CORFO 13CEE2-21592, Conicyt
  PAI Networks (REDES170151), and Project PLU180009, 2018.
\end{acks}

%
% The next two lines define the bibliography style to be used, and the bibliography file.
\bibliographystyle{ACM-Reference-Format}
\bibliography{locweb}

\end{document}
